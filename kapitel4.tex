%
%	Praxisbezug
%

\pagebreak
\section{Awareness-Team}

\onehalfspacing

\subsection{Implementation}

To support the Code of Conduct and assist with conflict resolution, many large events and organizations employ Awareness-Teams.\footnote{See \textit{FFF (2021)}: Awareness OG Leitfaden. \cite{fffAware}}

The roles and responsibilities of the Awareness-Team are laid out in an Awareness-Concept. This concept should be agreed upon in advance and preferably enjoys majority support.

Responsibilities could include the following tasks:

\begin{itemize}
    \item Code of conduct enforcement
    \item Safe space maintenance
    \item Incident response at gatherings
    \item Accessibility and inclusion support
    \item Harassment and discrimination prevention
    \item Anti-bullying initiatives
    \item Sexual assault prevention
\end{itemize}

The role of the team is twofold: it should create Awareness before problems occur and respond to incidents after they happen.

A key capability for the team member is the ability to de-escalate a conflict situation by communicating effectively and nonviolently.

The effectiveness of an Awareness-Team depends heavily on organizational support, a clear mandate, appropriate training, and genuine commitment of the organization to creating safer, more inclusive environments.

\subsection{Real-World Example \& Safe Word}

One of the first event agencies to embrace Awareness for their concerts and festivals was \href{https://fkpscorpio.com/}{FKP Scorpio}.

They ensured that Awareness-Teams were on-site and visible, and also introduced Safe Spaces as safe environments for anyone seeking help.

To further support this, and make access even simpler, FKP Scorpio introduced a code phrase, "Where is Panama?", that would signal staff that the person needs immediate support, without alerting bystanders or the perpetrator.\footnote{See \textit{FKP Scorpio (2023)}: Wo geht's nach Panama. \cite{panama}}

Rumours have it that the phrase was chosen as a reference to the children's book "The Trip to Panama" by Janosch.

Over the years, this concept has spread through most events. 

A similar idea is behind the international Signal for Help that alerts to violence at home.

\subsection{Conclusion}

The mindset and practice of Non-Violent Communication are key capabilities for anyone involved in Awareness Work.

A deep understanding of the shared humanity of all of us, the knowledge that all living beings strive for happiness, will make sure that the Awareness-Concept is sound and that the work of the Awareness-Team will be successful and beneficial.
