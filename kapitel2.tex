%
%	Begrifflichkeiten
%

\pagebreak
\section{Nonviolent Communication}

\onehalfspacing

\subsection{Philosophical Background}

Let's have a look at Marshall Rosenberg's philosophical background.

Rosenberg was heavily influenced by Carl Rogers' person-centered approach, particularly the belief in people's inherent capacity for empathy and growth. This shapes NVC's emphasis on creating conditions for authentic human connection.

The term "nonviolent" directly references Gandhi's concept of ahimsa (non-harm). Rosenberg adopted Gandhi's belief that violence stems from thinking that makes enemies of people, and that actual change comes through compassionate understanding rather than force.

\subsection{Concepts}

NVC assumes humans are naturally compassionate and that violence (physical or verbal) is a learned behavior that occurs when we don't know how to express our needs effectively.

Drawing from humanistic psychology, NVC assumes that all humans share the same fundamental needs (e.g., safety, autonomy, connection, meaning, etc.), though we may have different strategies for meeting them.

Influenced by general semantics, Rosenberg also believed that how we use language shapes our consciousness and relationships. "Life-alienating" language creates separation, while "life-serving" language fosters connection.

NVC rejects punishment-based thinking and instead emphasizes understanding the needs and feelings behind actions. This reflects a more restorative than retributive approach to conflict.

\subsection{Jackal and Giraffe}

In NVC, Marshall Rosenberg used the metaphors of a Jackal and a Giraffe to illustrate two fundamentally different ways of communicating and thinking.

Rosenberg chose the Jackal as a symbol for life-alienating communication because it's a scavenger, representing communication that "feeds on" judgment and criticism.

For life-serving communication, he chose the giraffe, because it has the largest heart of any land animal and can see far because of its height, representing communication from the heart with a broader perspective.

NVC teaches people to recognize when they're in "jackal mode" and consciously shift to "giraffe consciousness" - moving from judgment to curiosity, from demands to requests, and from blame to understanding needs.

\subsection{Communication Steps}

The four steps of Nonviolent Communication form the core process for expressing yourself and receiving others empathically. 

These steps are:

\begin{enumerate}
    \item Observation without evaluation. Observations are what we see or hear that we identify as the stimulus to our reactions
    \item Feelings. Feelings represent our emotional experience and physical sensations associated with our needs that have been met or that remain unmet
    \item Needs. Our needs are an expression of our most profound shared humanity, our core values, and our most profound human longings
    \item Request. We make requests to assess how likely we are to get cooperation for strategies we have in mind for meeting our needs
\end{enumerate}

As an example, this is how it could sound in a real-world example:

\begin{itemize}
    \item When I see dirty dishes left in the sink overnight (observation)
    \item I feel frustrated (feeling) 
    \item because I need order and cooperation in our shared space (need). 
    \item Would you be willing to either wash your dishes after eating or let me know if you'd like help creating a dish-washing schedule that works for both of us? (request)
\end{itemize}

\subsection{Enhancements}

Together with \href{https://www.adventusart.de/}{Adventus Art} we added two more steps to the process:

\begin{enumerate}
    \setcounter{enumi}{5}
    \item Personal responsibility. We offer to take personal responsibility for our needs
    \item Appreciation. We appreciate the person and the situation
\end{enumerate}
