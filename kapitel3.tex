%
%	Theorieteil
%

\pagebreak
\section{Conflicts}

\onehalfspacing

\subsection{Definition}

Conflict as a social phenomenon is a complex aspect of human interaction that extends far beyond individual disagreements.

Generally speaking, a conflict emerges when two or more parties perceive their interests, values, needs, or goals as incompatible or mutually exclusive. It's an inherent feature of social life, arising wherever people interact.

Potential structural sources of social conflict are:

\begin{itemize}
    \item Resource competition
    \item (Gender) Role conflicts
    \item Value differences
    \item Power imbalances\footnote{See \textit{Omelaenko, N. (2021)}: Conflict As A Social Phenomenon. \cite{conflict}}
\end{itemize}

\subsection{Scenarios}

In social groups, gatherings, or events, there are a couple of possible conflict scenarios:

\begin{itemize}
    \item Miscommunication
    \item Disagreements
    \item Inappropriate/harassing behavior
    \begin{itemize}
        \item Offensive pictures (e.g., in online meetings)
        \item Unwanted (physical) contact
        \item Misogyny and other forms of hate speech
        \item Mansplaining
        \item Any other forms of (White) male dominance behavior
    \end{itemize}    
\end{itemize}

For the remainder of this paper, we will focus on inappropriate or harassing behaviour.

\subsection{Code of Conduct}

A key step in dealing with conflict is to define the boundaries of acceptable behavior up front clearly.

The most common way is to create a clear and unambiguous Code of Conduct.\footnote{See \textit{Ruby Berlin e.V. (2017)}: Berlin Code of Conduct. \cite{coc}}

The Code of Conduct should cover:

\begin{itemize}
    \item A clear and detailed definition of expected behaviour, beyond the "Be excellent to each other"
    \item A clear and detailed definition of unacceptable behavior, such as unwanted contact or unsolicited communication
    \item The consequences of unacceptable behavior, such as removal from the event
    \item Instructions on where and how to address grievances
\end{itemize}

Most large social events, such as conferences or festivals, now have a clear Code of Conduct and set boundaries on acceptable behavior.

\subsection{Awareness Team}

But even with a Code of Conduct defined, conflicts will arise.

To deal with conflicts, Awareness-Teams could be created. In the next chapter, we will explore the concept of Awareness-Teams and consider what a team might look like.

To prevent and resolve conflicts, the Awareness-Teams themselves must be able to engage and communicate without resorting to violence. The practice of nonviolent communication is an excellent basis for the work of Awareness-Teams.
