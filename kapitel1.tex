%
%	Einfuehrung
%

\pagebreak
\section{Introduction}

\onehalfspacing

\subsection{Non-Violent Communication}

Nonviolent Communication (NVC) is a communication process created by psychologist Marshall B. Rosenberg.

Rosenberg believes that every communication, both verbal and nonverbal, is a form of exchange and negotiation between partners and that we can perform these exchanges with or without compassion.

NVC assumes that compassionate communication yields different results than uncompassionate communication and that these differences have a significant impact on both individual and societal levels.\footnote{See \textit{Rosenberg, M. (2015)}: Nonviolent Communication. \cite{rosenbergNvc}}

Or, as Paul Watzlawik put it, "One cannot not communicate." \footnote{\textit{Watzlawik, P. (1972)}: Pragmatics of Human Communication. \cite{watzlawick}}

\subsection{Awareness}

Awareness in the context of a social setting means being attentive to situations where a person's boundaries and sense of security are crossed,

Awareness work is based on the understanding that spaces are created differently by people who are in them. We always want to treat each other respectfully so that everyone feels as safe as possible, and we want to be attentive and sensitive to individual boundaries and needs.

Awareness work is always partisan, and boundary crossings are defined by those affected themselves.\footnote{See \textit{Fluid (2022)}: Fluid Awareness Konzept. \cite{fluidAware}}

\subsection{Research Question \& Method}

This paper will examine the use of NVC for conflict resolution. To do this, we will perform a Case Study and evaluate the Awareness concept in activist settings.\footnote{See \textit{McCombes, S. (2019)}: What is a Case Study. \cite{caseScribbr}}

The goal of this paper is to establish whether NVC is a valuable tool for conflict resolution and awareness in group settings.

\subsection{Gender-neutral Pronouns}

Our society is becoming more open, inclusive, and gender-fluid, and now I think it's time to think about using gender-neutral pronouns in scientific texts, too. Two well-known researchers, Abigail C. Saguy and Juliet A. Williams, both from UCLA, propose to use the singular they/them instead: "The universal singular they is inclusive of people who identify as male, female or nonbinary." \footnote{\textit{Saguy, A. (2020)}: Why We Should All Use They/Them Pronouns. \cite{pronouns}} The aim is to support an inclusive approach in science through gender-neutral language. 

In this paper, I'll attempt to follow this suggestion and invite all my readers to do the same for future articles. Thank you!

If you're not sure about the definitions of gender and sex and how to use them, have a look at the definitions\footnote{See \textit{APA (2021)}: Definitions Related to Sexual Orientation. \cite{apaDefinitions}} by the American Psychological Association.

\subsection{Climate Emergency}

As Professor Rahmstorf puts it: "Without immediate, decisive climate protection measures, my children currently attending high school could already experience a 3-degree warmer Earth. No one can say exactly what this world would look like—it would be too far outside the entire experience of human history. But almost certainly, this earth would be full of horrors for the people who would have to experience it." \footnote{\textit{Rahmstorf, A. (2024)}: Climate and Weather at 3 Degrees More. \cite{3dgreesMore}}
